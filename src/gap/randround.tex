\subsection{Randomized rounding}

Directly using randomized rounding with GAP is cumbersome, since the constraint
\[ \sum_{i \in \delta(j)} x_{i,j} = 1 \quad \forall j \in J \]
need not be satisfied. As a result, some jobs may not be scheduled
and some jobs will be scheduled on multiple machines.

We'll use a variant of randomized rounding which is strictly better than randomized rounding:
instead of rounding each $x_e$ independently, we'll assign each job independently.
For job $j$, we'll assign it to machine $i$ with probability $x_{i,j}$.
So if $X$ is the rounded solution, then $[X_{i,j}: i \in \delta(j)]$
follows the Multinoulli distribution with $\Pr(X_{i,j} = 1) = x_{i,j}^*$.
Since $\sum_{i \in \delta(j)} x_{i,j}^* = 1$, this is possible and each job
will be assigned to a single machine.
\[ \textrm{Expected load on machine } i
= \E\left(\sum_{j \in \delta(i)} p_{i,j}X_{i,j}\right)
= \sum_{j \in \delta(i)} p_{i,j}x_{i,j}^*
\le T_i \]

In the linear term $\sum_{j \in \delta(i)} p_{i,j}X_{i,j}$,
the variables correspond to different jobs, so they are independent.
Therefore, we can use a variant of Bernstein's inequality
(theorem \ref{thm:lin-bound-bern-conc}) to prove concentration.
\[ \Pr[\textrm{load on machine } i \ge (1+\delta)T_i]
\le \exp\left( -\frac{1}{2} \frac{q_i}{T_i} \frac{1}{1+\frac{q_i}{3T_i}} \right) \]
When $q_i$ can be as high as $T_i$, we get
\[ \Pr[\textrm{load on machine } i \ge 2T_i]
\le e^{-\frac{3}{8}} \le 0.6873 \]
This is poor concentration.

We'll soon see that sub-isotripic rounding guarantees that load on machine $i$
is at most $T_i + q_i/(1-2\delta)$, for a parameter $\delta$. For randomized rounding we get
\[ \Pr\left[\textrm{load on machine } i \ge T_i + \frac{q_i}{1-2\delta} \right]
\le \exp\left( -\frac{1}{2} \frac{q_i}{T_i} \frac{1}{(1-2\delta)(\frac{4}{3}-2\delta)}\right) \]
This is poor concentration, especially when $q_i$ is small compared to $T_i$.
