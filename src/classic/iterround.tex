\subsection{Iterative rounding}

We'll only consider problems whose LP relaxations are of this form:
\[ \min_{x \in \mathbb{R}^n} c^Tx \textrm{ where } Ax \le b \wedge 0 \le x \le 1 \]

Iterative rounding is a deterministic rounding algorithm
where in each iteration, we round one or more variables to 0 or 1.

We first obtain an optimal solution $x^*$ to the LP. Let $x^{(0)} = x^*$.
Then there are several iterations.
In iteration $i$, we modify $x^{(i-1)}$ according to certain rules to obtain $x^{(i)}$.
Call this modification algorithm $\iterround$, i.e. $x^{(i)} = \iterround(x^{(i-1)})$.
Unlike $\randround$, the design of $\iterround$ may rely on the structure of the problem.

$\iterround$ must follows these 3 rules:
\begin{itemize}
\item $\iterround(x)$ should be feasible for the LP.
\item values of integer-valued variables remain the same, i.e.
\[ x^{(i)}_j \in \{0, 1\} \implies x^{(i+1)}_j = x^{(i)}_j \]
\item the number of integer-valued variables increases by at least 1.
\end{itemize}
Therefore, in at most $n$ iterations, we obtain an integral solution $X$.

Ideally, we'll want $c^Tx^{(i)}$ to not increase in each iteration,
i.e. design $\iterround$ so that it satisfies this property: $c^T\iterround(x) \le c^Tx$.
It may be difficult to do so, and it is impossible if the integrality gap is not 1.
To make it easier to design such an $\iterround$,
we apply one (or both) of these tricks before invoking $\iterround$:
\begin{itemize}
\item \textbf{Relaxation}:
We relax a carefully-chosen set of constraints,
i.e. change $(Ax)_i \le b_i$ to $(Ax)_i \le (1+\epsilon)b_i$.
We also ensure that a constraint is never relaxed twice.
This means that when the algorithm ends, all constraints are approximately satisfied.
Usually we drop a constraint altogether if we can prove that
dropping it won't violate it too much.
\item \textbf{Rounding}:
We round variables with values above a certain threshold to 1,
i.e. set $x_j$ to 1 if $x_j \ge \alpha$, such that the resulting LP is still feasible.
If we can ensure that $c^T\iterround(x) \le c^Tx$ when each fractional variable
is less than $\alpha$, then when the algorithm ends,
the objective value is at most $1/\alpha$ times the optimal.
\end{itemize}
We will ignore rounding in this document; we'll only consider relaxation.

We now look at 2 common strategies for designing $\iterround$.

\subsubsection{Strategy 1: obtaining extreme-point solution to residual LP}

Treat integral variables as constants to get a new LP, called the residual LP,
and obtain an optimal extreme-point solution to it.
This will give us a feasible solution to the old LP with the same or better
objective value and whose integral variables have the same value as before.
The only task that remains is to prove that the number of integral variables increases.
This is done by using the rank lemma on the residual LP.
The proof is problem-specific and relies on the structure of the problem.

\begin{theorem}[Rank lemma]
Given an optimal extreme point solution,
the maximum number of linearly independent tight constraints
equals the number of variables.
\end{theorem}

\subsubsection{Strategy 2: finding constraints of low dimensionality}
\label{sec:it-str-2}

\begin{definition}
Let $Ax \le b$ be the constraints in an LP other than the $0 \le x \le 1$ constraints,
where $A$ is an $m$ by $n$ matrix. Let $R \subseteq [n]$.
Let $w_i$ be the vector of coefficients of $R$ in the $i^{\textrm{th}}$ constraint,
i.e. $w_i = [A[i, j]: j \in R]$. Let $W = \{w_i: 1 \le i \le m\}$.
Then $W$ is called the alive component of the constraint matrix
$A$ with respect to $R$ and is denoted as $A[R]$.
\end{definition}

The strategy for designing $\iterround$ is as follows:
Carefully choose a subset $R$ of fractional variables
(usually $R$ is the set of all the fractional variables).
Let $W = A[R]$. Then we prove that $\dim(W) < \abs{R}$.
Choosing $R$ and proving $\dim(W) < \abs{R}$ is problem-specific
and relies on the structure of the problem.

$\dim(W) < \abs{R}$ means that the nullspace of $W$ has positive dimension,
so there is a non-zero vector $z \in \mathbb{R}^{\abs{R}}$ such that $Wz = 0$.
Define $y \in \mathbb{R}^n$ as
\[ y_j = \begin{cases} z_j & \textrm{if } j \in R
\\ 0 & \textrm{if } j \not\in R \end{cases} \]
If $c^Ty > 0$, replace $z$ by $-z$ and $y$ by $-y$.

Let $x^{(i+1)} = x^{(i)} + \gamma y$, where
$\gamma$ is the maximum value such that $0 \le x^{(i+1)} \le 1$.
Since $0 \le x^{(i)} \le 1$ and all variables in $R$ are fractional, $\gamma > 0$.
\begin{itemize}
\item Given how we choose $\gamma$, $0 \le x^{(i+1)} \le 1$.
$Wz = 0 \implies Ay = 0 \implies Ax^{(i+1)} = Ax^{(i)} \le b$.
Therefore, $x^{(i+1)}$ is feasible for the LP.
\item Since $R$ only contains fractional variables,
the values of integral variables don't change.
\item Since we're choosing $\gamma$ to be the maximum possible,
$\exists j \in R, x^{(i+1)}_j \in \{0, 1\}$.
This means that the number of integral variables increases.
\item $\gamma > 0$ and $c^Ty \le 0$ implies that the objective value does not increase.
\end{itemize}

The above properties are exactly what we need for successfully applying iterative rounding.

When $\dim(W) = \abs{R}-1$, then there is a unique (up to scaling) non-zero $z$
that satisfies $Wz = 0$. However, if $\dim(W)$ is even smaller,
then we have more freedom on how to choose $z$.
It may be possible to choose $z$ intelligently so that the final integral solution
satisfies additional desirable properties.
In fact, this is the basis of sub-isotropic rounding.

\subsubsection{What is it good for?}

Iterative rounding is useful in scenarios where there are `hard constraints' in the LP,
i.e. whose violation cannot be tolerated.
For example, for the maximum-weight bipartite matching problem,
the constraints ensure that the solution is a matching.
For this problem, a solution that is not a matching is unacceptable.
