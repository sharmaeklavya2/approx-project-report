\usepackage{amsmath,amstext,amssymb,amsfonts}
\usepackage{amsthm}

\newtheorem{theorem}{Theorem}
\newtheorem{definition}{Definition}
\newtheorem{example}{Example}
\newtheorem{corollary}{Corollary}[theorem]
\newtheorem{lemma}[theorem]{Lemma}
\newtheorem{Claim}[theorem]{Claim}	

\newcommand*{\floor}[1]{\left\lfloor #1 \right\rfloor}
\newcommand*{\ceil}[1]{\left\lceil #1 \right\rceil}
\newcommand*{\abs}[1]{\left\lvert #1 \right\rvert}
\newcommand*{\norm}[1]{\left\lVert #1 \right\rVert}
\newcommand*{\Th}{^{\textrm{th}}}
\DeclareMathOperator*{\trace}{trace}
\DeclareMathOperator*{\diag}{diag}
\DeclareMathOperator*{\Var}{Var}
\DeclareMathOperator*{\argmin}{argmin}
\DeclareMathOperator*{\argmax}{argmax}
\newcommand*{\E}{\operatorname{\mathbb{E}}}
\newcommand*{\randround}{\operatorname{randround}}
\newcommand*{\iterround}{\operatorname{iterround}}
\newcommand*{\optprog}[3]{
\begin{array}{*3{>{\displaystyle}l}}
#1 & \multicolumn{2}{>{\displaystyle}l}{#2}
#3 \end{array}}

% parentheses
\newcommand{\paren}[1]{\left(#1 \right )}
\newcommand{\Paren}[1]{\left(#1 \right )}

% linear algebra
\newcommand{\inprod}[1]{\left\langle #1\right\rangle}

% norm
\newcommand{\snorm}[1]{\norm{#1}^2}

% L2 norm
\newcommand{\normt}[1]{\norm{#1}_{\scriptstyle 2}}
\newcommand{\snormt}[1]{\norm{#1}^2_2}

% set braces
\newcommand{\set}[1]{\left\{#1\right\}}
\newcommand{\Set}[1]{\left\{#1\right\}}
